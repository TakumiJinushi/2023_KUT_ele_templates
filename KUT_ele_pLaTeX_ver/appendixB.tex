\chapter{日本語文献を用いる場合}
このテンプレートからも分かるように,日本語文献のタイトルは太字となってしまう.
これは,日本語の文献がBib{\LaTeX}側で考慮されていないことによるエラーである.
そこで,日本語でもIEEEスタイルで出力できるjIEEE.bstを使えば,タイトルが太字にならない.
しかし,標準ではjIEEEスタイルは\TeX Liveで同梱されていないのでダウンロードする必要がある~\cite{jieeetran}.
以下,jIEEEによる参考文献のスタイル変更の方法を記載する.

\begin{enumerate}
    \item jIEEEをGitHub~\cite{jieeetran}からダウンロードしてZIPファイルを解凍する.
    \item jIEEEtran-master/jIEEEtran にあるjIEEEtran.bstを作業フォルダ(main.texのあるフォルダ)に入れる.
    \item main.texで次の箇所を編集する.
        \begin{enumerate}
            \item \textbackslash usepackage[...]\{biblatex\} を \textbackslash usepackage\{cite\} に変更する.
            \item \textbackslash addbibresource\{references.bib\} を\% でコメントアウトする.
            \item \textbackslash printbibliography[title={参考文献}] を\% でコメントアウトする.
            \item 直前でコメントアウトした行の次の行で \textbackslash bibliography\{references\} と記述する.
            \item 更に,次の行で \textbackslash bibliographystyle\{jIEEEtran\} と記述する.
        \end{enumerate}
    \item 一度 \LaTeX ファイルをコンパイルする(まだ上手くいかない箇所があるはずだが).
    \item references.bib で次の箇所を編集する.
        \begin{enumerate}
            \item 作成したPDFで大文字が小文字で表示されたり,日本語の著者名が省略された場合,
                references.bib で該当箇所を$1$重波括弧\{\}から$2$重波括弧\{\{\}\}にする.\\
                (例. \{Accessed on Oct. 18, 2023\} を \{\{Accessed on Oct. 18, 2023\}\},\\
                    \{奥村晴彦, 黒木裕介\}を\{\{\{奥村晴彦, 黒木裕介\}\}\}に変更)
            \item 著書の一部を記載している場合,バッククォート二つとダブルクォート二つで書名を囲う.\\
                (例.[改定第$9$版]\LaTeXe 美文書作成入門が,``[改定第$9$版]\LaTeXe 美文書作成入門''となるはず)
        \end{enumerate}
\end{enumerate}

上記の手順を踏んで編集すれば,より高知工科大学 電子系の
「卒業研究報告書(学士論文)・修士論文の執筆要項」に近づく.